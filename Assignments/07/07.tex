\documentclass[]{article}
\usepackage{lmodern}
\usepackage{amssymb,amsmath}
\usepackage{ifxetex,ifluatex}
\usepackage{fixltx2e} % provides \textsubscript
\ifnum 0\ifxetex 1\fi\ifluatex 1\fi=0 % if pdftex
  \usepackage[T1]{fontenc}
  \usepackage[utf8]{inputenc}
\else % if luatex or xelatex
  \ifxetex
    \usepackage{mathspec}
  \else
    \usepackage{fontspec}
  \fi
  \defaultfontfeatures{Mapping=tex-text,Scale=MatchLowercase}
  \newcommand{\euro}{€}
\fi
% use upquote if available, for straight quotes in verbatim environments
\IfFileExists{upquote.sty}{\usepackage{upquote}}{}
% use microtype if available
\IfFileExists{microtype.sty}{%
\usepackage{microtype}
\UseMicrotypeSet[protrusion]{basicmath} % disable protrusion for tt fonts
}{}
\makeatletter
\@ifpackageloaded{hyperref}{}{%
\ifxetex
  \usepackage[setpagesize=false, % page size defined by xetex
              unicode=false, % unicode breaks when used with xetex
              xetex]{hyperref}
\else
  \usepackage[unicode=true]{hyperref}
\fi
}
\@ifpackageloaded{color}{
    \PassOptionsToPackage{usenames,dvipsnames}{color}
}{%
    \usepackage[usenames,dvipsnames]{color}
}
\makeatother
\hypersetup{breaklinks=true,
            bookmarks=true,
            pdfauthor={},
            pdftitle={Assignment 7},
            colorlinks=true,
            citecolor=blue,
            urlcolor=blue,
            linkcolor=magenta,
            pdfborder={0 0 0}
            }
\urlstyle{same}  % don't use monospace font for urls
\usepackage{color}
\usepackage{fancyvrb}
\newcommand{\VerbBar}{|}
\newcommand{\VERB}{\Verb[commandchars=\\\{\}]}
\DefineVerbatimEnvironment{Highlighting}{Verbatim}{commandchars=\\\{\}}
% Add ',fontsize=\small' for more characters per line
\newenvironment{Shaded}{}{}
\newcommand{\KeywordTok}[1]{\textcolor[rgb]{0.00,0.44,0.13}{\textbf{{#1}}}}
\newcommand{\DataTypeTok}[1]{\textcolor[rgb]{0.56,0.13,0.00}{{#1}}}
\newcommand{\DecValTok}[1]{\textcolor[rgb]{0.25,0.63,0.44}{{#1}}}
\newcommand{\BaseNTok}[1]{\textcolor[rgb]{0.25,0.63,0.44}{{#1}}}
\newcommand{\FloatTok}[1]{\textcolor[rgb]{0.25,0.63,0.44}{{#1}}}
\newcommand{\ConstantTok}[1]{\textcolor[rgb]{0.53,0.00,0.00}{{#1}}}
\newcommand{\CharTok}[1]{\textcolor[rgb]{0.25,0.44,0.63}{{#1}}}
\newcommand{\SpecialCharTok}[1]{\textcolor[rgb]{0.25,0.44,0.63}{{#1}}}
\newcommand{\StringTok}[1]{\textcolor[rgb]{0.25,0.44,0.63}{{#1}}}
\newcommand{\VerbatimStringTok}[1]{\textcolor[rgb]{0.25,0.44,0.63}{{#1}}}
\newcommand{\SpecialStringTok}[1]{\textcolor[rgb]{0.73,0.40,0.53}{{#1}}}
\newcommand{\ImportTok}[1]{{#1}}
\newcommand{\CommentTok}[1]{\textcolor[rgb]{0.38,0.63,0.69}{\textit{{#1}}}}
\newcommand{\DocumentationTok}[1]{\textcolor[rgb]{0.73,0.13,0.13}{\textit{{#1}}}}
\newcommand{\AnnotationTok}[1]{\textcolor[rgb]{0.38,0.63,0.69}{\textbf{\textit{{#1}}}}}
\newcommand{\CommentVarTok}[1]{\textcolor[rgb]{0.38,0.63,0.69}{\textbf{\textit{{#1}}}}}
\newcommand{\OtherTok}[1]{\textcolor[rgb]{0.00,0.44,0.13}{{#1}}}
\newcommand{\FunctionTok}[1]{\textcolor[rgb]{0.02,0.16,0.49}{{#1}}}
\newcommand{\VariableTok}[1]{\textcolor[rgb]{0.10,0.09,0.49}{{#1}}}
\newcommand{\ControlFlowTok}[1]{\textcolor[rgb]{0.00,0.44,0.13}{\textbf{{#1}}}}
\newcommand{\OperatorTok}[1]{\textcolor[rgb]{0.40,0.40,0.40}{{#1}}}
\newcommand{\BuiltInTok}[1]{{#1}}
\newcommand{\ExtensionTok}[1]{{#1}}
\newcommand{\PreprocessorTok}[1]{\textcolor[rgb]{0.74,0.48,0.00}{{#1}}}
\newcommand{\AttributeTok}[1]{\textcolor[rgb]{0.49,0.56,0.16}{{#1}}}
\newcommand{\RegionMarkerTok}[1]{{#1}}
\newcommand{\InformationTok}[1]{\textcolor[rgb]{0.38,0.63,0.69}{\textbf{\textit{{#1}}}}}
\newcommand{\WarningTok}[1]{\textcolor[rgb]{0.38,0.63,0.69}{\textbf{\textit{{#1}}}}}
\newcommand{\AlertTok}[1]{\textcolor[rgb]{1.00,0.00,0.00}{\textbf{{#1}}}}
\newcommand{\ErrorTok}[1]{\textcolor[rgb]{1.00,0.00,0.00}{\textbf{{#1}}}}
\newcommand{\NormalTok}[1]{{#1}}
\setlength{\parindent}{0pt}
\setlength{\parskip}{6pt plus 2pt minus 1pt}
\setlength{\emergencystretch}{3em}  % prevent overfull lines
\providecommand{\tightlist}{%
  \setlength{\itemsep}{0pt}\setlength{\parskip}{0pt}}
\setcounter{secnumdepth}{0}

\usepackage{listings} % Required for insertion of code
\title{Assignment 7}
\date{}

% Redefines (sub)paragraphs to behave more like sections
\ifx\paragraph\undefined\else
\let\oldparagraph\paragraph
\renewcommand{\paragraph}[1]{\oldparagraph{#1}\mbox{}}
\fi
\ifx\subparagraph\undefined\else
\let\oldsubparagraph\subparagraph
\renewcommand{\subparagraph}[1]{\oldsubparagraph{#1}\mbox{}}
\fi
% HERGET
\input{/home/herget/LMU/templates/ans-env.tex}

\lstset{basicstyle={\ttfamily},literate={↓}{$\downarrow{}$}{1}}
 % END HERGET
\begin{document}
\maketitle

\textbf{C++ Programming Course, Summer Term 2018}

\begin{quote}
``Because I'm hard, you will not like me.\\
\hspace*{0.333em}But the more you hate me, the more you will learn.''\\
-- \href{https://www.youtube.com/watch?v=3j3_iPskjxk}{Gunnery
Sgt.~Hartman}\\
\hspace*{0.333em}\hspace*{0.333em}\hspace*{0.333em}(gunned down by one
of his recruits a few weeks later, so \ldots{} because of a private)
\end{quote}

\section{7-1 Fixing the Google Tech
Interview}\label{fixing-the-google-tech-interview}

Google has an official YouTube channel for recruiting software engineers
where they published this video of a staged tech interview, presumably
to demonstrate what Google would expect from candidates:

\url{https://www.youtube.com/watch?v=XKu_SEDAykw}

For reasons one can only speculate about, comments have been disabled
for this video.

Wipe the floor with the C++ implementation that is showcased as an
acceptable solution\\
(at \href{https://youtu.be/XKu_SEDAykw?t=15m29s}{15:29} in the video):

\begin{enumerate}
\def\labelenumi{\arabic{enumi}.}
\tightlist
\item
  Shake your head in disbelief and disgust
    %\problemAnswer{
        \begin{lstlisting}
        (╯°□°)╯︵ ┻━┻
        ┬─┬ ノ( ゜-゜ノ)
        ┻━┻ ︵ヽ(`Д´)ノ︵ ┻━┻
        \end{lstlisting}
%    }
\item
  Provide an improved variant of the \texttt{HasPairWithSum} function\\
   (feel free to also improve its name)
\item
  Provide use cases to explain in which numerous ways the original
  implementation is lacking and why your improvements are essential\\
   (add insult to injury: discuss parallelization aspects in particular)

   \problemAnswer{
        UseCases:
        \begin{itemize}
            \item Numbers are not int
            \item Numbers are not stored in a Vector
            \item No parallelization is implemented -> large number of items take very long :(
        \end{itemize}
        Improvements:
        \begin{itemize}
            \item Templated function which works with ranges and any suitable type / container
        \end{itemize}
   }
\item
  Become aware that you got pretty good at C++
\end{enumerate}

\section{7-2 Act Like You Got a Pair}\label{act-like-you-got-a-pair}

The utility template \texttt{std::pair\textless{}A,B\textgreater{}} is a
useful companion due to its utmost simplicity and predictability: just a
struct of two values \texttt{first} and \texttt{second} with independent
types.

Experiment with minimal examples to answer the following questions:

\begin{itemize}
\tightlist
\item
  How do comparison operators of
  \texttt{pair\textless{}A,B\textgreater{}} depend on types \texttt{A}
  and \texttt{B}?
\item
  How can we specify comparison of
  \texttt{pair\textless{}A0,B0\textgreater{}} and
  \texttt{pair\textless{}A1,B1\textgreater{}}?
\item
  What is the benefit of \texttt{std::get} over \texttt{pair.first} /
  \texttt{pair.second}\\
   (use compiler explorer)
\item
  Why don't we just always use \texttt{std::get}?
\end{itemize}

The most traditional ways to represent a range are:

\begin{itemize}
\tightlist
\item
  a pair of iterators
\item
  a pair of an iterator and an offset
\end{itemize}

Implement as minimal proof-of-concepts:

\begin{itemize}
\tightlist
\item
  Specialize \texttt{std::make\_pair} for iterator and offset, that
  is:\\
   Define a variant of \texttt{std::make\_pair(Iter,int)} which only
  matches parameter types \texttt{Iter} that provide a type definition
  \texttt{iterator\_category}
\item
  Define

  \begin{itemize}
  \tightlist
  \item
    \texttt{std::begin(std::pair\textless{}Iter,int\textgreater{}\ p)}\\
     returning \texttt{std::get\textless{}0\textgreater{}(p)}
  \item
    \texttt{std::end(std::pair\textless{}Iter,int\textgreater{}\ p)}\\
     returning
    \texttt{std::advance(std::get\textless{}0\textgreater{}(p),\ std::get\textless{}1\textgreater{}(p))}
  \end{itemize}
\end{itemize}

Use compiler explorer to check that your specializations have no
overhead.

A usage example:

\begin{Shaded}
\begin{Highlighting}[]
\NormalTok{std::vector<std::string> v \{}
   \StringTok{"Monday"}\NormalTok{, }\StringTok{"Tuesday"}\NormalTok{, }\StringTok{"Wednesday"}\NormalTok{,}
   \StringTok{"Thursday"}\NormalTok{, }\StringTok{"Brainfryday"}\NormalTok{, }\StringTok{"Saturday"}\NormalTok{, }\StringTok{"Sunday"} \NormalTok{\};}

\KeywordTok{auto} \NormalTok{range = std::make_pair(v.begin(), }\DecValTok{6}\NormalTok{);}

\KeywordTok{for} \NormalTok{(}\DataTypeTok{const} \KeywordTok{auto} \NormalTok{& value : range) \{}
   \NormalTok{std::cout << value << }\StringTok{'}\CharTok{\textbackslash{}n}\StringTok{'}\NormalTok{;}
\NormalTok{\}}
\end{Highlighting}
\end{Shaded}

\section{7-3 Iterate, Evaluate, Destroy}\label{iterate-evaluate-destroy}

References:

\begin{itemize}
\tightlist
\item
  \url{http://en.cppreference.com/w/cpp/algorithm}
\end{itemize}

\subsection{7-3-1 Algorithm Categories}\label{algorithm-categories}

Like container categories (sequential, associative, wrapper), STL
algorithms are categorized into conceptual groups.

Two of those are \emph{Modifying Sequence Operation} and
\emph{Non-Modifying Sequence Operation}.

\begin{itemize}
\tightlist
\item
  In the \href{http://en.cppreference.com/w/cpp/algorithm}{overview of
  STL algorithms}, algorithms like \texttt{std::sort} and
  \texttt{std::partition} are not in die \emph{Modifying} category. Why?

  \problemAnswer{
  \texttt{std::sort} and  \texttt{std::partition} are in the \texttt{Partitioning / Sorting operations}.
  }
\item
  Are there algorithms that return their result as a new sequence? Why?
  Discuss a minimal use-case to illustrate this.
\end{itemize}

\subsection{7-3-2 Iterator Invalidation}\label{iterator-invalidation}

\begin{itemize}
\tightlist
\item
  Which algorithms allow to add or remove elements from their input
  ranges?
\item
  Explain how \texttt{std::list}, \texttt{std::deque},
  \texttt{std::vector} and \texttt{std::map} differ in iterator
  invalidation rules.\\
   Also discuss the differences between iterator invalidation rules for
  erasure (removing container elements) and insertion (adding elements).
\end{itemize}

\section{7-4 Runtimes / They are
a-Changin'}\label{runtimes-they-are-a-changin}

\subsection{7-4-0 Prerequisites}\label{prerequisites}

Clone \emph{Celero} from \url{https://github.com/DigitalInBlue/Celero}
and experiment with the examples in the distribution.

\subsection{7-4-1 Shaming Virtual}\label{shaming-virtual}

Let's assume a colleague of yours uses virtual for virtually everything.

\begin{itemize}
\item
  Implement micro-benchmarks using Celero that demonstrate the
  disadvantages of \texttt{virtual} (runtime polymorphism) in the most
  drastic way you can.
\item
  Evaluate performance of CRTP vs.~virtual in a micro-benchmark.\\
   You can use the CRTP iterator base classes from your solution to
  assignment 6 if you want.
\end{itemize}

\end{document}
