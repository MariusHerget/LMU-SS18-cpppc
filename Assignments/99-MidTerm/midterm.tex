\documentclass[]{article}
\usepackage{lmodern}
\usepackage{amssymb,amsmath}
\usepackage{ifxetex,ifluatex}
\usepackage{fixltx2e} % provides \textsubscript
\ifnum 0\ifxetex 1\fi\ifluatex 1\fi=0 % if pdftex
  \usepackage[T1]{fontenc}
  \usepackage[utf8]{inputenc}
\else % if luatex or xelatex
  \ifxetex
    \usepackage{mathspec}
  \else
    \usepackage{fontspec}
  \fi
  \defaultfontfeatures{Mapping=tex-text,Scale=MatchLowercase}
  \newcommand{\euro}{€}
\fi
% use upquote if available, for straight quotes in verbatim environments
\IfFileExists{upquote.sty}{\usepackage{upquote}}{}
% use microtype if available
\IfFileExists{microtype.sty}{%
\usepackage{microtype}
\UseMicrotypeSet[protrusion]{basicmath} % disable protrusion for tt fonts
}{}
\makeatletter
\@ifpackageloaded{hyperref}{}{%
\ifxetex
  \usepackage[setpagesize=false, % page size defined by xetex
              unicode=false, % unicode breaks when used with xetex
              xetex]{hyperref}
\else
  \usepackage[unicode=true]{hyperref}
\fi
}
\@ifpackageloaded{color}{
    \PassOptionsToPackage{usenames,dvipsnames}{color}
}{%
    \usepackage[usenames,dvipsnames]{color}
}
\makeatother
\hypersetup{breaklinks=true,
            bookmarks=true,
            pdfauthor={},
            pdftitle={Midterm Assignment},
            colorlinks=true,
            citecolor=blue,
            urlcolor=blue,
            linkcolor=magenta,
            pdfborder={0 0 0}
            }
\urlstyle{same}  % don't use monospace font for urls
\usepackage{color}
\usepackage{fancyvrb}
\newcommand{\VerbBar}{|}
\newcommand{\VERB}{\Verb[commandchars=\\\{\}]}
\DefineVerbatimEnvironment{Highlighting}{Verbatim}{commandchars=\\\{\}}
% Add ',fontsize=\small' for more characters per line
\newenvironment{Shaded}{}{}
\newcommand{\KeywordTok}[1]{\textcolor[rgb]{0.00,0.44,0.13}{\textbf{{#1}}}}
\newcommand{\DataTypeTok}[1]{\textcolor[rgb]{0.56,0.13,0.00}{{#1}}}
\newcommand{\DecValTok}[1]{\textcolor[rgb]{0.25,0.63,0.44}{{#1}}}
\newcommand{\BaseNTok}[1]{\textcolor[rgb]{0.25,0.63,0.44}{{#1}}}
\newcommand{\FloatTok}[1]{\textcolor[rgb]{0.25,0.63,0.44}{{#1}}}
\newcommand{\ConstantTok}[1]{\textcolor[rgb]{0.53,0.00,0.00}{{#1}}}
\newcommand{\CharTok}[1]{\textcolor[rgb]{0.25,0.44,0.63}{{#1}}}
\newcommand{\SpecialCharTok}[1]{\textcolor[rgb]{0.25,0.44,0.63}{{#1}}}
\newcommand{\StringTok}[1]{\textcolor[rgb]{0.25,0.44,0.63}{{#1}}}
\newcommand{\VerbatimStringTok}[1]{\textcolor[rgb]{0.25,0.44,0.63}{{#1}}}
\newcommand{\SpecialStringTok}[1]{\textcolor[rgb]{0.73,0.40,0.53}{{#1}}}
\newcommand{\ImportTok}[1]{{#1}}
\newcommand{\CommentTok}[1]{\textcolor[rgb]{0.38,0.63,0.69}{\textit{{#1}}}}
\newcommand{\DocumentationTok}[1]{\textcolor[rgb]{0.73,0.13,0.13}{\textit{{#1}}}}
\newcommand{\AnnotationTok}[1]{\textcolor[rgb]{0.38,0.63,0.69}{\textbf{\textit{{#1}}}}}
\newcommand{\CommentVarTok}[1]{\textcolor[rgb]{0.38,0.63,0.69}{\textbf{\textit{{#1}}}}}
\newcommand{\OtherTok}[1]{\textcolor[rgb]{0.00,0.44,0.13}{{#1}}}
\newcommand{\FunctionTok}[1]{\textcolor[rgb]{0.02,0.16,0.49}{{#1}}}
\newcommand{\VariableTok}[1]{\textcolor[rgb]{0.10,0.09,0.49}{{#1}}}
\newcommand{\ControlFlowTok}[1]{\textcolor[rgb]{0.00,0.44,0.13}{\textbf{{#1}}}}
\newcommand{\OperatorTok}[1]{\textcolor[rgb]{0.40,0.40,0.40}{{#1}}}
\newcommand{\BuiltInTok}[1]{{#1}}
\newcommand{\ExtensionTok}[1]{{#1}}
\newcommand{\PreprocessorTok}[1]{\textcolor[rgb]{0.74,0.48,0.00}{{#1}}}
\newcommand{\AttributeTok}[1]{\textcolor[rgb]{0.49,0.56,0.16}{{#1}}}
\newcommand{\RegionMarkerTok}[1]{{#1}}
\newcommand{\InformationTok}[1]{\textcolor[rgb]{0.38,0.63,0.69}{\textbf{\textit{{#1}}}}}
\newcommand{\WarningTok}[1]{\textcolor[rgb]{0.38,0.63,0.69}{\textbf{\textit{{#1}}}}}
\newcommand{\AlertTok}[1]{\textcolor[rgb]{1.00,0.00,0.00}{\textbf{{#1}}}}
\newcommand{\ErrorTok}[1]{\textcolor[rgb]{1.00,0.00,0.00}{\textbf{{#1}}}}
\newcommand{\NormalTok}[1]{{#1}}
\setlength{\parindent}{0pt}
\setlength{\parskip}{6pt plus 2pt minus 1pt}
\setlength{\emergencystretch}{3em}  % prevent overfull lines
\providecommand{\tightlist}{%
  \setlength{\itemsep}{0pt}\setlength{\parskip}{0pt}}
\setcounter{secnumdepth}{0}

\title{Midterm Assignment}
\date{}

% Redefines (sub)paragraphs to behave more like sections
\ifx\paragraph\undefined\else
\let\oldparagraph\paragraph
\renewcommand{\paragraph}[1]{\oldparagraph{#1}\mbox{}}
\fi
\ifx\subparagraph\undefined\else
\let\oldsubparagraph\subparagraph
\renewcommand{\subparagraph}[1]{\oldsubparagraph{#1}\mbox{}}
\fi
% HERGET
\input{/home/herget/LMU/templates/ans-env.tex}
 % END HERGET
\begin{document}
\maketitle

\section{1. Types, Classes, Objects and their
State}\label{types-classes-objects-and-their-state}

\subsection{1.1 Object Definition}\label{object-definition}

Assuming you are confronted with a class that supports the following use
case:

\begin{Shaded}
\begin{Highlighting}[]
\NormalTok{DateParser date_parser;}
\NormalTok{date_parser.set_date(today);}

\KeywordTok{auto} \NormalTok{day = date_parser.day_of_week();}
\end{Highlighting}
\end{Shaded}

What is your objection to this? How would you change the interface of
class \texttt{DateParser}?

\problemAnswer{
Possible improvements:
    \begin{enumerate}
        \item Konstruktoren, in denen man direkt die Zeit setzen kann
        \item \texttt{set\_date} stellt nicht klar welche Formate es annimmt (entweder alle oder genauer spezifizieren)
        \item \texttt{day\_of\_week()} mit Uebergabeparametern
    \end{enumerate}
}

\subsection{1.2 Standard Object
Semantics}\label{standard-object-semantics}

Given the following implementation of class \texttt{SemanticSurprise}:

\begin{Shaded}
\begin{Highlighting}[]
\KeywordTok{class} \NormalTok{SemanticSurprise \{}
\KeywordTok{public}\NormalTok{:}

  \NormalTok{SemanticSurprise(}\DataTypeTok{int} \NormalTok{value)}
  \NormalTok{: _value(value) \{ \}}

  \NormalTok{SemanticSurprise(}\DataTypeTok{const} \NormalTok{SemanticSurprise & other)}
  \NormalTok{: _value(other._value) \{ \}}

  \DataTypeTok{bool} \KeywordTok{operator}\NormalTok{==(}\DataTypeTok{const} \NormalTok{SemanticSurprise & other) }\DataTypeTok{const} \NormalTok{\{}
    \KeywordTok{return} \NormalTok{_value == other._value;}
  \NormalTok{\}}

  \DataTypeTok{int} \NormalTok{value() \{}
    \KeywordTok{return} \NormalTok{_value;}
  \NormalTok{\}}

\KeywordTok{private}\NormalTok{:}
  \DataTypeTok{int} \NormalTok{_value;}
\NormalTok{\};}
\end{Highlighting}
\end{Shaded}

\begin{itemize}
\tightlist
\item
  Are there compiler errors? If so: why?
  \problemAnswer{
    Nope: \url{https://godbolt.org/g/ihDPeV}
  }
\item
  Name the method/operation on \texttt{SemanticSurprise} in every line
  marked with \texttt{Op?}, including those that lead to compiler
  errors.
\item
  Which values are returned in the lines marked with \texttt{value?}
  (given the line does not lead to a compiler error). Explain why these
  values are returned.
\item
  How can this implementation be simplified? What is the requirement for
  this simplification?
  \problemAnswer{
    Copy Constructor can be defaulted.
  }

\end{itemize}
\begin{Shaded}
\begin{Highlighting}[]
\NormalTok{SemanticSurprise s1;          }\CommentTok{// Default Constructor}
\NormalTok{s1.value();                   }\CommentTok{// Compiler Error (No matching constructor)}

\NormalTok{SemanticSurprise s2(}\DecValTok{4}\NormalTok{);       }\CommentTok{// Parameterized (Assignment) Constructor}
\NormalTok{s2.value();                   }\CommentTok{// 4 --> constructor defined}

\NormalTok{SemanticSurprise s3 = s2;     }\CommentTok{// Copy-assignment Constructor}
\NormalTok{s3.value();                   }\CommentTok{// 4 --> assigned in s2}

\NormalTok{SemanticSurprise s4(s3);      }\CommentTok{// Copy Constructor}
\NormalTok{s4 == s3;                     }\CommentTok{// True --> Copy constructor fullfill the \texttt{operator==}}
\NormalTok{s2 != s3;                     }\CommentTok{// Compiler Error (invailid operands to binary expression)}
\end{Highlighting}
\end{Shaded}

Given the implementation of \texttt{SemanticSurprise} above, the
following use case leads to a compiler error:

\begin{Shaded}
\begin{Highlighting}[]
\DataTypeTok{void} \NormalTok{print_surprise(}\DataTypeTok{const} \NormalTok{SemanticSurprise & s) \{}
  \NormalTok{std::cout << }\StringTok{"SemanticSurprise.value: "} \NormalTok{<< s.value() << std::endl;}
\NormalTok{\}}

\NormalTok{print_surprise(SemanticSurprise(}\DecValTok{10}\NormalTok{));}
\end{Highlighting}
\end{Shaded}

\begin{itemize}
\tightlist
\item
  Explain the compiler error and how the implementation of
  \texttt{SemanticSurprise} has to be corrected.
  \problemAnswer{
    The given SemanticSurprise is marked as const but the function \texttt{value()} is not marked as const. Therefore it could potentially modify the object. Possible solutions:
        \begin{itemize}
            \item Mark \texttt{value()} as \texttt{const}
            \item Implement a second function: \\\texttt{int value() const \{ return \_value; \}}
        \end{itemize}
  }
\end{itemize}

\subsection{1.3 Resource Ownership}\label{resource-ownership}

Consider the following use case:

\begin{Shaded}
\begin{Highlighting}[]
\CommentTok{// ...}
\KeywordTok{if} \NormalTok{(condition) \{}
        \NormalTok{LogFileReader log_reader(}\StringTok{"performance.log"}\NormalTok{);}
        \KeywordTok{if} \NormalTok{(log_reader.lines().size() > }\DecValTok{1024}\NormalTok{) \{ }\KeywordTok{return}\NormalTok{; \}}
\NormalTok{\}}
\CommentTok{// ...}
\end{Highlighting}
\end{Shaded}

\begin{itemize}
\tightlist
\item
  Judging from its usage illustrated above: which rule must be satisfied
  in the implementation of class \texttt{LogFileReader} (or one of its
  members)?
  \problemAnswer{
  ???

        Potential memory leak when the if condition throws an error (or is it? Since lifetime of the variable is ended auf the mother if condition).\\[2em]
??? Which rule!\\[2em]

        In addition \texttt{log\_reader.lines()} should be constant since other applications can still write log information to the file (this saves us a mutex / sharable resource).
  }
\item
  Name a popular technique in resource management that depends on this
  rule, and briefly explain its principle.
  \problemAnswer{???}
\end{itemize}

\section{2. Iterators}\label{iterators}

\section{2.1 Algorithm Basics}\label{algorithm-basics}

The following algorithm dereferences and returns a given iterator's
successor unless the successor's referenced value matches some
condition. The algorithm is semantically correct but does not compile
for iterators of some containers.

\begin{itemize}
\tightlist
\item
  Which containers? Briefly explain why the algorithm does not work for
  these and how it has to be changed.
\end{itemize}

\begin{Shaded}
\begin{Highlighting}[]
\KeywordTok{template} \NormalTok{<}\KeywordTok{typename} \NormalTok{Iterator, }\KeywordTok{typename} \NormalTok{Value = }\KeywordTok{typename} \NormalTok{Iterator::value_type>}
\NormalTok{Value next_value_or_default(Iterator it, Iterator end, Value }\KeywordTok{default}\NormalTok{) \{}
  \KeywordTok{if} \NormalTok{(it == end || it}\DecValTok{+1} \NormalTok{== end) \{ }\KeywordTok{return} \KeywordTok{default}\NormalTok{; \}}
  \KeywordTok{return} \NormalTok{it[}\DecValTok{1}\NormalTok{];}
\NormalTok{\}}
\end{Highlighting}
\end{Shaded}


\problemAnswer{
    First it does not compile. \texttt{default} is in some compilers a reserved keyword. Renamed it to \texttt{dfault}.\\

    The problem of this implementation is the increment of the iterator \texttt{it+1}. This is only suitable for \texttt{RandomAccessIterators}. The correct implementation should be \texttt{it++} to be also suitable for \texttt{ForwardIterator}, \texttt{BidirectionalIterator} and \texttt{InputIterator}.\\ Therefore all container using the \texttt{RandomAccessIterator Concept} should work with this implementation.

    \url{https://godbolt.org/g/8hrmHb}
}

\section{2.2 Container Wrapper}\label{container-wrapper}

The STL's \texttt{std::vector} guarantess that its elements are stored
in a contiguous memory region and is therefore compatible to C-style
arrays. The member function \texttt{.data()} returns a pointer to the
vector's underlying raw memory.

For many hardware-tuning techniques, data is accessed in chunks.
Assuming a \texttt{std::vector\textless{}uint32\_t\textgreater{}} and 64
bytes per cache line, for example, vector elements could be loaded in
chunks of 64/(32/8) = 16 elements.

\begin{itemize}
\item
  Write a container wrapper

\begin{verbatim}
cpppc::chunks<B, T, Container>
\end{verbatim}

  that provides a sequential container interface on elements in
  \texttt{Container} in chunks of maximum size \texttt{B} bytes.
\end{itemize}

The template signature is derived from \texttt{std::stack}, a similar
wrapper type defined in the STL:

\url{http://en.cppreference.com/w/cpp/container/stack}

You only need to implement a minimal interface like constructors and
\texttt{begin()}, \texttt{end()}, however.

Example use case:

\begin{Shaded}
\begin{Highlighting}[]
\NormalTok{std::vector<}\DataTypeTok{uint16_t}\NormalTok{> v_us;}

\CommentTok{//                                    ,-- wrapped container}
\CommentTok{//                                    |}
\NormalTok{cpppc::chunks<}\DecValTok{128}\NormalTok{, uint16t_t, std::vector<}\DataTypeTok{uint16_t}\NormalTok{>> v_chunks(v_us);}
\CommentTok{//             |      |}
\CommentTok{//             |      '-- element type}
\CommentTok{//             |}
\CommentTok{//             '-- maximum size of a}
\CommentTok{//                 single chunk in bytes}

\CommentTok{// Iterate chunks:}
\KeywordTok{auto} \NormalTok{first_chunk = v_chunks.begin();}
\KeywordTok{auto} \NormalTok{chunk_size  = std::distance(v_chunks.begin(), v_chunks.end());}
                   \CommentTok{// --> 128/(16/8) = 64}

\CommentTok{// Iterators on elements in a chunk:}
\DataTypeTok{uint16_t} \NormalTok{first_chunk_elem = *first_chunk.begin();}
\DataTypeTok{uint16_t} \NormalTok{third_chunk_elem = first_chunk[}\DecValTok{2}\NormalTok{];}

\CommentTok{// Pointer to data in second chunk:}
\DataTypeTok{uint16_t} \NormalTok{chunk_1_data = v_chunks[}\DecValTok{1}\NormalTok{].data();}
\CommentTok{// Pointer to data in third chunk (= end pointer of data in second chunk):}
\DataTypeTok{uint16_t} \NormalTok{chunk_2_data = v_chunks[}\DecValTok{2}\NormalTok{].data();}
\end{Highlighting}
\end{Shaded}

Note that you need two iterator types:

\begin{itemize}
\tightlist
\item
  \texttt{cpppc::chunks\textless{}...\textgreater{}::iterator} for
  iterating chunks; it references
  \texttt{cpppc::chunks\textless{}...\textgreater{}::value\_type} (=
  type of single chunks)
\item
  and another iterator type to access chunk elements.
\end{itemize}

It might be best if you use existing STL containers like
\texttt{std::vector\textless{}T\textgreater{}} or
\texttt{std::array\textless{}T,\ ChunkSize\textgreater{}} as chunk type:
these would already provide the sequence container interface to chunk
elements you need.

\textbf{Note}

The \texttt{cpppc::chunks} container adapter can (and should) be
implemented as a \emph{view}, that is: without copying elements of the
underlying vector.

\subsection{2.X Bonus}\label{x-bonus}

If you want to implement more sophisticated type size calculations like
alignment requirements, refer to the C++ type support utilities:

\url{http://en.cppreference.com/w/cpp/types}

Or make yourself familiar with the boost Align library (brace for
impact):

\url{http://www.boost.org/doc/libs/1_63_0/doc/html/align.html}

\section{3. Algorithms, Function Templates, Type
Deduction}\label{algorithms-function-templates-type-deduction}

\begin{itemize}
\item
  Implement a function interface \texttt{void\ log10(X)} that accepts a
  numeric value of type \texttt{X} and:

  \begin{itemize}
  \tightlist
  \item
    prints the base 10 logarithm of the value if it is an integer
  \item
    prints the base 10 logarithm of the value's square root if it is a
    floating point value
  \end{itemize}
\item
  Implement a function interface
  \texttt{void\ print\_walk(T\ begin,\ T\ end)} that accepts a range of
  iterators of type \texttt{T} and prints all values in the range. In
  this, the iteration order depends on the iterator type:

  \begin{itemize}
  \tightlist
  \item
    for random-access iterators, the order should be (pseudo) random but
    every element in the range must only be printed once
    \problemAnswer{Is the implementation random enough?}
  \item
    for input iterators, elements are printed in order from
    \texttt{begin} to \texttt{end-1}
  \item
    for bidirectional iterators, elements are printed in reverse order
    from \texttt{end-1} to \texttt{begin}
  \end{itemize}
\end{itemize}

\section{4. Thread-Safety}\label{thread-safety}

\subsection{4.1 Parallelism and STL
Containers}\label{parallelism-and-stl-containers}

Given the following operations on an instance of \texttt{std::vector},
consider operations in the same table row to be executed by multiple
threads in parallel:

\problemAnswer{All mentioning of rules (e.g. \texttt{Rule 1.}) are based on \url{https://en.cppreference.com/w/cpp/container#Thread_safety}}

\begin{Shaded}
\begin{Highlighting}[]
\CommentTok{// Shared vector instance accessed by thread A and thread B:}
\NormalTok{std::vector<}\DataTypeTok{int}\NormalTok{> v;}

\CommentTok{// thread A:                    | thread B:}
\CommentTok{// =============================|==============================}
   \NormalTok{std::vector<}\DataTypeTok{int}\NormalTok{> a;          | std::vector<}\DataTypeTok{int}\NormalTok{> b;}
\CommentTok{// Safe: no shared resources (Rule 1.)}
\CommentTok{// -----------------------------+------------------------------}
   \DataTypeTok{int} \NormalTok{xa = v[}\DecValTok{3}\NormalTok{];               | }\DataTypeTok{int} \NormalTok{xb = v[}\DecValTok{4}\NormalTok{];}
\CommentTok{// Safe: Rad-only -> Rule 2.}
\CommentTok{// -----------------------------+------------------------------}
   \NormalTok{v[}\DecValTok{3}\NormalTok{] = }\DecValTok{123}\NormalTok{;                  | v[}\DecValTok{4}\NormalTok{] = }\DecValTok{345}\NormalTok{;}
\CommentTok{// Safe: Rule 3.}
\CommentTok{// -----------------------------+------------------------------}
   \NormalTok{v[}\DecValTok{3}\NormalTok{] = }\DecValTok{123}\NormalTok{;                  | }\DataTypeTok{int} \NormalTok{xb = v[}\DecValTok{3}\NormalTok{];}
\CommentTok{// Not safe: Modyfing and reading the same element in the same}
\CommentTok{// container can result in ambigious behaviour}
\CommentTok{// -----------------------------+------------------------------}
   \NormalTok{v.push_back(}\DecValTok{24}\NormalTok{);             | v.size();}
\CommentTok{// Not safe: The underlying container is modified in A and its}
\CommentTok{// properties (which change) is read in B -> ambigious behaviour}
\CommentTok{// -----------------------------+------------------------------}
   \NormalTok{v.back();                    | v.push_back(}\DecValTok{54}\NormalTok{);}
\CommentTok{// Not safe: Rule 2. is violated. }
\CommentTok{// -----------------------------+------------------------------}
   \NormalTok{v.begin();                   | v.push_back(}\DecValTok{34}\NormalTok{);}
\CommentTok{// Not 100\% safe: push_back can invailidate the iteratos}
\CommentTok{// -----------------------------+------------------------------}
   \NormalTok{v.back();                    | v.pop_back();}
\CommentTok{// Not safe: The underlying container is modified and read at the}
\CommentTok{// same time \& pop_back can invailidate the iterators}
\CommentTok{// -----------------------------+------------------------------}
\end{Highlighting}
\end{Shaded}

\begin{itemize}
\tightlist
\item
  For every pair of operation on the same table row, give a brief
  explanation on the guarantees with respect to thread-safety according
  to the C++ standard.
\end{itemize}

\subsection{4.2 Producer-Consumer
Problem}\label{producer-consumer-problem}

Simply put, one thread is producing goods and another thread is
consuming goods. We want the consumer thread to wait \textbf{using a
condition variable}, and we want \texttt{goods.push(i)} to be mutually
exclusive to \texttt{goods.pop()}.

We are letting c++ and c-- be surrogates for this mutual exclusion,
since we can easily check if we correctly end up with 0 in the end.

Run the code as it is, and you will see that the net value is way off:

\begin{quote}
\texttt{Note:}

I know there is an easier solution, however the condition variable
approach is more efficient in many cases and not as commonly known as it
should be.
\end{quote}

\begin{Shaded}
\begin{Highlighting}[]
\OtherTok{#include <iostream>}
\OtherTok{#include <thread>}
\OtherTok{#include <condition_variable>}
\OtherTok{#include <mutex>}
\OtherTok{#include <chrono>}
\OtherTok{#include <queue>}
\KeywordTok{using} \KeywordTok{namespace} \NormalTok{std;}

\DataTypeTok{int} \NormalTok{main() \{}
    \DataTypeTok{int} \NormalTok{c = }\DecValTok{0}\NormalTok{;}
    \DataTypeTok{bool} \NormalTok{done = }\KeywordTok{false}\NormalTok{;}
    \NormalTok{queue<}\DataTypeTok{int}\NormalTok{> goods;}

    \NormalTok{thread producer([&]() \{}
        \KeywordTok{for} \NormalTok{(}\DataTypeTok{int} \NormalTok{i = }\DecValTok{0}\NormalTok{; i < }\DecValTok{500}\NormalTok{; ++i) \{}
            \NormalTok{goods.push(i);}
            \NormalTok{c++;}
        \NormalTok{\}}

        \NormalTok{done = }\KeywordTok{true}\NormalTok{;}
    \NormalTok{\});}

    \NormalTok{thread consumer([&]() \{}
        \KeywordTok{while} \NormalTok{(!done) \{}
            \KeywordTok{while} \NormalTok{(!goods.empty()) \{}
                \NormalTok{goods.pop();}
                \NormalTok{c--;}
            \NormalTok{\}}
        \NormalTok{\}}
    \NormalTok{\});}

    \NormalTok{producer.join();}
    \NormalTok{consumer.join();}
    \NormalTok{cout << }\StringTok{"Net: "} \NormalTok{<< c << endl;}
\NormalTok{\}}
\end{Highlighting}
\end{Shaded}
\end{document}
