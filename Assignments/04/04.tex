\documentclass[]{article}
\usepackage{lmodern}
\usepackage{amssymb,amsmath}
\usepackage{ifxetex,ifluatex}
\usepackage{fixltx2e} % provides \textsubscript
\ifnum 0\ifxetex 1\fi\ifluatex 1\fi=0 % if pdftex
  \usepackage[T1]{fontenc}
  \usepackage[utf8]{inputenc}
\else % if luatex or xelatex
  \ifxetex
    \usepackage{mathspec}
  \else
    \usepackage{fontspec}
  \fi
  \defaultfontfeatures{Mapping=tex-text,Scale=MatchLowercase}
  \newcommand{\euro}{€}
\fi
% use upquote if available, for straight quotes in verbatim environments
\IfFileExists{upquote.sty}{\usepackage{upquote}}{}
% use microtype if available
\IfFileExists{microtype.sty}{%
\usepackage{microtype}
\UseMicrotypeSet[protrusion]{basicmath} % disable protrusion for tt fonts
}{}
\makeatletter
\@ifpackageloaded{hyperref}{}{%
\ifxetex
  \usepackage[setpagesize=false, % page size defined by xetex
              unicode=false, % unicode breaks when used with xetex
              xetex]{hyperref}
\else
  \usepackage[unicode=true]{hyperref}
\fi
}
\@ifpackageloaded{color}{
    \PassOptionsToPackage{usenames,dvipsnames}{color}
}{%
    \usepackage[usenames,dvipsnames]{color}
}
\makeatother
\hypersetup{breaklinks=true,
            bookmarks=true,
            pdfauthor={},
            pdftitle={Assignment 4: Iterators, Function Templates},
            colorlinks=true,
            citecolor=blue,
            urlcolor=blue,
            linkcolor=magenta,
            pdfborder={0 0 0}
            }
\urlstyle{same}  % don't use monospace font for urls
\usepackage{color}
\usepackage{fancyvrb}
\newcommand{\VerbBar}{|}
\newcommand{\VERB}{\Verb[commandchars=\\\{\}]}
\DefineVerbatimEnvironment{Highlighting}{Verbatim}{commandchars=\\\{\}}
% Add ',fontsize=\small' for more characters per line
\newenvironment{Shaded}{}{}
\newcommand{\KeywordTok}[1]{\textcolor[rgb]{0.00,0.44,0.13}{\textbf{{#1}}}}
\newcommand{\DataTypeTok}[1]{\textcolor[rgb]{0.56,0.13,0.00}{{#1}}}
\newcommand{\DecValTok}[1]{\textcolor[rgb]{0.25,0.63,0.44}{{#1}}}
\newcommand{\BaseNTok}[1]{\textcolor[rgb]{0.25,0.63,0.44}{{#1}}}
\newcommand{\FloatTok}[1]{\textcolor[rgb]{0.25,0.63,0.44}{{#1}}}
\newcommand{\ConstantTok}[1]{\textcolor[rgb]{0.53,0.00,0.00}{{#1}}}
\newcommand{\CharTok}[1]{\textcolor[rgb]{0.25,0.44,0.63}{{#1}}}
\newcommand{\SpecialCharTok}[1]{\textcolor[rgb]{0.25,0.44,0.63}{{#1}}}
\newcommand{\StringTok}[1]{\textcolor[rgb]{0.25,0.44,0.63}{{#1}}}
\newcommand{\VerbatimStringTok}[1]{\textcolor[rgb]{0.25,0.44,0.63}{{#1}}}
\newcommand{\SpecialStringTok}[1]{\textcolor[rgb]{0.73,0.40,0.53}{{#1}}}
\newcommand{\ImportTok}[1]{{#1}}
\newcommand{\CommentTok}[1]{\textcolor[rgb]{0.38,0.63,0.69}{\textit{{#1}}}}
\newcommand{\DocumentationTok}[1]{\textcolor[rgb]{0.73,0.13,0.13}{\textit{{#1}}}}
\newcommand{\AnnotationTok}[1]{\textcolor[rgb]{0.38,0.63,0.69}{\textbf{\textit{{#1}}}}}
\newcommand{\CommentVarTok}[1]{\textcolor[rgb]{0.38,0.63,0.69}{\textbf{\textit{{#1}}}}}
\newcommand{\OtherTok}[1]{\textcolor[rgb]{0.00,0.44,0.13}{{#1}}}
\newcommand{\FunctionTok}[1]{\textcolor[rgb]{0.02,0.16,0.49}{{#1}}}
\newcommand{\VariableTok}[1]{\textcolor[rgb]{0.10,0.09,0.49}{{#1}}}
\newcommand{\ControlFlowTok}[1]{\textcolor[rgb]{0.00,0.44,0.13}{\textbf{{#1}}}}
\newcommand{\OperatorTok}[1]{\textcolor[rgb]{0.40,0.40,0.40}{{#1}}}
\newcommand{\BuiltInTok}[1]{{#1}}
\newcommand{\ExtensionTok}[1]{{#1}}
\newcommand{\PreprocessorTok}[1]{\textcolor[rgb]{0.74,0.48,0.00}{{#1}}}
\newcommand{\AttributeTok}[1]{\textcolor[rgb]{0.49,0.56,0.16}{{#1}}}
\newcommand{\RegionMarkerTok}[1]{{#1}}
\newcommand{\InformationTok}[1]{\textcolor[rgb]{0.38,0.63,0.69}{\textbf{\textit{{#1}}}}}
\newcommand{\WarningTok}[1]{\textcolor[rgb]{0.38,0.63,0.69}{\textbf{\textit{{#1}}}}}
\newcommand{\AlertTok}[1]{\textcolor[rgb]{1.00,0.00,0.00}{\textbf{{#1}}}}
\newcommand{\ErrorTok}[1]{\textcolor[rgb]{1.00,0.00,0.00}{\textbf{{#1}}}}
\newcommand{\NormalTok}[1]{{#1}}
\usepackage{longtable,booktabs}
\setlength{\parindent}{0pt}
\setlength{\parskip}{6pt plus 2pt minus 1pt}
\setlength{\emergencystretch}{3em}  % prevent overfull lines
\providecommand{\tightlist}{%
  \setlength{\itemsep}{0pt}\setlength{\parskip}{0pt}}
\setcounter{secnumdepth}{0}

\title{Assignment 4: Iterators, Function Templates}
\date{}

% Redefines (sub)paragraphs to behave more like sections
\ifx\paragraph\undefined\else
\let\oldparagraph\paragraph
\renewcommand{\paragraph}[1]{\oldparagraph{#1}\mbox{}}
\fi
\ifx\subparagraph\undefined\else
\let\oldsubparagraph\subparagraph
\renewcommand{\subparagraph}[1]{\oldsubparagraph{#1}\mbox{}}
\fi

% HERGET
\usepackage{tcolorbox}
\tcbuselibrary{theorems, breakable, skins}
\newtcbtheorem{example}% environment name
              {Answer}% Title text
  {enhanced, % tcolorbox styles
  attach boxed title to top center={yshift=-2.5mm},
  colback=white, colframe=black, colbacktitle=white, coltitle=black,
  boxed title style={size=small,colframe=white},
  fonttitle=\bfseries,
  sharp corners=all,
  breakable
  }%
  {ex}% label prefix
 % END HERGET

\begin{document}
\maketitle

\textbf{C++ Programming Course, Summer Term 2018}

\href{/assignment-04/a-04/}{Source Code}

\section{4-0: Watch these Talks}\label{watch-these-talks}

A talk at ACCU has just been published on YouTube that summarizes the
relevance of STL algorithms and explains the mental model behind them
really well:

\begin{itemize}
\tightlist
\item
  \href{https://www.youtube.com/watch?v=bXkWuUe9V2I}{Jonathan Boccara:
  105 STL Algorithms in Less Than an Hour}
\end{itemize}

Everything in this talk agrees with the discussion of STL algorithms in
our lab sessions, and the presentation is stellar.

\begin{quote}
There is a trap that lies right at the beginning of the path to STL
algorithms, and this trap is called for\_each.
\end{quote}

In another talk, Marshall Clow gives an introduction and some hints on
how to implement good STL algorithm abstractions:

\begin{itemize}
\tightlist
\item
  \href{https://www.youtube.com/watch?v=h4Jl1fk3MkQ}{Marshall Clow: STL
  Algorithms - why you should use them, and how to write your own}
\end{itemize}

\section{4-1: Algorithms / Function
Templates}\label{algorithms-function-templates}

In the following, you will implement two algorithms as function
templates.

Your implementations should be a combination of functions provided by
the STL and must not contain explicit loops like \texttt{for} or
\texttt{while}.

You may use any algorithm interface defined in the C++14 standard. For
example, you may use \texttt{std::minmax} (since C++11) in your
implementations, but not \texttt{std::for\_each\_n} (C++17).

References:

\begin{itemize}
\tightlist
\item
  C++ Algorithm Library:
  \url{http://en.cppreference.com/w/cpp/algorithm}
\item
  C++ Iterator Library:
  \url{http://en.cppreference.com/w/cpp/header/iterator}
\end{itemize}

\subsection{4-1-1: find\_mean\_rep}\label{findux5fmeanux5frep}

Implement a function template \texttt{cpppc::find\_mean\_rep} which
accepts a range specified by two iterators in the \texttt{InputIterator}
category and returns an iterator to the element in the range that is
closest to the mean of the range.

Your implementation should be automatically more efficient for random
access iterator ranges without specialization.

\textbf{Function interface:}

\begin{Shaded}
\begin{Highlighting}[]
\KeywordTok{template} \NormalTok{<}\KeywordTok{typename} \NormalTok{InputIter>}
\NormalTok{InputIter find_mean_rep(InputIter first, InputIter last);}
\end{Highlighting}
\end{Shaded}

\textbf{Example:}

\begin{Shaded}
\begin{Highlighting}[]
\NormalTok{std::vector<}\DataTypeTok{int}\NormalTok{> v \{ }\DecValTok{1}\NormalTok{, }\DecValTok{2}\NormalTok{, }\DecValTok{3}\NormalTok{, }\DecValTok{30}\NormalTok{, }\DecValTok{40}\NormalTok{, }\DecValTok{50} \NormalTok{\}; }\CommentTok{// mean: 21}
\KeywordTok{auto} \NormalTok{closest_to_mean_it = cpppc::find_mean_rep(v.begin(), v.end());}
\CommentTok{// -> iterator at index 3 (|21-30| = 9)}
\end{Highlighting}
\end{Shaded}

\subsection{4-1-2: histogram
(RandomAccessIterator)}\label{histogram-randomaccessiterator}

Implement a function template \texttt{cpppc::histogram} which accepts a
range specified by two iterators in the \texttt{InputIterator} category
and \textbf{replaces} each value by the number of its occurrences in the
range.

Note that values of the input range are modified.

Values are unique in the result histogram and ordered by their first
occurrence in the range. The function returns an iterator past the final
element in the histogram.

Only integer values (int, long, size\_t, \ldots{}) have to be supported.

\textbf{Function interface:}

\begin{Shaded}
\begin{Highlighting}[]
\KeywordTok{template} \NormalTok{<}\KeywordTok{typename} \NormalTok{RAIt>}
\NormalTok{RAIt histogram(RAIt first, RAIt last);}
\end{Highlighting}
\end{Shaded}

\textbf{Example:}

\begin{Shaded}
\begin{Highlighting}[]
\NormalTok{std::vector<}\DataTypeTok{int}\NormalTok{> v \{ }\DecValTok{1}\NormalTok{, }\DecValTok{5}\NormalTok{, }\DecValTok{5}\NormalTok{, }\DecValTok{3}\NormalTok{, }\DecValTok{4}\NormalTok{, }\DecValTok{1}\NormalTok{, }\DecValTok{5}\NormalTok{, }\DecValTok{7} \NormalTok{\};}
\KeywordTok{auto} \NormalTok{hist_end = cpppc::histogram(v.begin(), v.end());}
\CommentTok{// -> 1: 2 occurences}
\CommentTok{//    5: 3 occurences}
\CommentTok{//    5: skipped}
\CommentTok{//    ...}
\CommentTok{// -> \{ 2, 3, 1, 1, 1 \}  <- occurrences}
\CommentTok{//      ^  ^  ^  ^  ^}
\CommentTok{//      |  |  |  |  |}
\CommentTok{//      1  5  3  4  7    <- value}
\CommentTok{//}
\CommentTok{// -> returns iterator at position 5}
\end{Highlighting}
\end{Shaded}

\section{4-2: Custom Containers /
Iterators}\label{custom-containers-iterators}

\subsection{4-2-1: Sparse Array}\label{sparse-array}

Implement a sparse array (see
\url{https://en.wikipedia.org/wiki/Sparse_array}) container that
satisfies the \texttt{std::array} interface.

A sparse array is a usually large array with most elements set to a
default value like 0. Instead of allocating memory for all values, only
non-default values are actually allocated.

Different from vectors, arrays are static containers so their size does
not change after instantiation. The values of its elements can be
iterated and changed, but elements cannot be added or removed.

References:

\begin{itemize}
\tightlist
\item
  std::array in the C++ Standard Template Library:
  \url{http://en.cppreference.com/w/cpp/container/array}
\end{itemize}

\begin{Shaded}
\begin{Highlighting}[]
\DataTypeTok{int} \KeywordTok{default} \NormalTok{= }\DecValTok{0}\NormalTok{;}
\NormalTok{cpppc::sparse_array<}\DataTypeTok{int}\NormalTok{, }\DecValTok{10000}\NormalTok{> sa(}\KeywordTok{default}\NormalTok{);}
\CommentTok{// no actual values in `sa` yet, all set to default}
\NormalTok{size_t sa_size = sa.size(); }\CommentTok{// -> 10000}
\NormalTok{assert(sa[}\DecValTok{345}\NormalTok{] == }\DecValTok{0}\NormalTok{);       }\CommentTok{// returns `default` if no value}
                            \CommentTok{// set at position}

\NormalTok{sa[}\DecValTok{230}\NormalTok{] = }\DecValTok{23}\NormalTok{;}
\NormalTok{assert(sa[}\DecValTok{230}\NormalTok{] == }\DecValTok{23}\NormalTok{);}

\NormalTok{sa[}\DecValTok{420}\NormalTok{] = }\DecValTok{42}\NormalTok{;}
\CommentTok{// two values stored in `sa`}

\CommentTok{// Must be compatible with STL algorithms:}

\KeywordTok{auto} \NormalTok{found_42 = std::find(sa.begin(), sa.end(), }\DecValTok{42}\NormalTok{);}
\CommentTok{// -> iterator at 'virtual' position 420 (the second 'real' value)}
\end{Highlighting}
\end{Shaded}

\textbf{Notes:}

\begin{itemize}
\tightlist
\item
  Your implementation has to detect assignment of a value reference. Try
  a temporary proxy:
  \url{https://en.wikibooks.org/wiki/More_C++_Idioms/Temporary_Proxy}.
\item
  Obviously, the \texttt{sparse\_array} cannot provide direct access to
  its underlying array (as it doesn't exist) so method \texttt{data()}
  is not provided.
\end{itemize}

\subsection{4-2-2: Lazy Sequence}\label{lazy-sequence}

Write a class template \texttt{lazy\_sequence} that implements the
sequence concept and is initialized with its size and a generator
function.

A lazy sequence does not store elements (unlike containers), instead a
generator function is used to create values only when they are accessed.
It is not modifiable, similar to \texttt{const} sequential containers.

We define the \texttt{Sequence} concept, for an instance \texttt{s} of
sequence type \texttt{S}:

\begin{longtable}[c]{@{}ll@{}}
\toprule
Type & Synopsis\tabularnewline
\midrule
\endhead
\texttt{S::value\_type} & Values in the sequence\tabularnewline
\texttt{S::iterator} & Iterator to \texttt{S::value\_type}, satisfies
\texttt{RandomAccessIterator}\tabularnewline
\texttt{S::size\_type} & Type to represent sequence size (length, number
of values)\tabularnewline
\bottomrule
\end{longtable}

\begin{longtable}[c]{@{}lll@{}}
\toprule
Expression & Returns & Synopsis\tabularnewline
\midrule
\endhead
\texttt{s.begin()} & \texttt{S::iterator} & Iterator on first
value\tabularnewline
\texttt{s.end()} & \texttt{S::const\_iterator} & Iterator past final
value\tabularnewline
\texttt{s.size()} & \texttt{S::size\_type} &
\texttt{s.end()\ -\ s.begin()}\tabularnewline
\texttt{s{[}i{]}} & \texttt{const\ S::value\_type\ \&} or
\texttt{S::value\_type} & \texttt{*(s.begin()\ +\ i)}\tabularnewline
\bottomrule
\end{longtable}

As usual, all types have value semantics (assignment, copy, comparison,
\ldots{}).

\textbf{Example:}

\begin{Shaded}
\begin{Highlighting}[]
  \NormalTok{lazy_sequence<}\DataTypeTok{int}\NormalTok{> seq(}\DecValTok{10}\NormalTok{,}
                        \NormalTok{[](}\DataTypeTok{int} \NormalTok{i) \{}
                          \KeywordTok{return} \NormalTok{(}\DecValTok{100} \NormalTok{+ i * i);}
                        \NormalTok{\});}

  \CommentTok{// `seq` does not contain any elements at this point, but it satisfies the}
  \CommentTok{// STL Sequence Container concept:}

  \NormalTok{std::cout << }\StringTok{"sequence size: "} \NormalTok{<< seq.size() << }\StringTok{'}\CharTok{\textbackslash{}n}\StringTok{'}\NormalTok{;}

  \KeywordTok{auto} \NormalTok{it = seq.begin();}
  \CommentTok{// still nothing computed}

  \DataTypeTok{int} \NormalTok{first = *it; }\CommentTok{// <-- returns value from generator(it._pos)}

  \KeywordTok{for} \NormalTok{(}\KeywordTok{auto} \NormalTok{e : seq) \{}
    \CommentTok{// generates values one by one:}
    \NormalTok{std::cout << e << }\StringTok{" "}\NormalTok{;}
  \NormalTok{\}}
  \NormalTok{std::cout << std::endl;}
  \CommentTok{// prints:}
  \CommentTok{// 100 101 104 109 116 125 136 149 164 181}
\end{Highlighting}
\end{Shaded}

Strictly speaking, there is no \emph{Sequence} concept in C++ or the
STL.\\
We will encounter the concept category \emph{Range} for this soon.

\section{4-X: Challenges}\label{x-challenges}

\subsection{4-X-1: Bresenham's Line
Sequence}\label{x-1-bresenhams-line-sequence}

Make yourself familiar with
\href{http://graphics.idav.ucdavis.edu/education/GraphicsNotes/Bresenhams-Algorithm.pdf}{Bresenham's
Line Algorithm} and implement it as sequential concept:

\begin{Shaded}
\begin{Highlighting}[]
\NormalTok{raster_image img(width, height);}

\NormalTok{raster_point p_from = img[\{x0, y0\}];}
\NormalTok{raster_point p_to   = img[\{x1, y1\}];}

\NormalTok{raster_line_seq = raster_line_sequence(raster_image, p_from, p_to);}

\KeywordTok{for} \NormalTok{(}\KeywordTok{auto} \NormalTok{line_point : raster_line_seq) \{}
  \NormalTok{img[line_point].color = black;}
\NormalTok{\}}
\end{Highlighting}
\end{Shaded}

\subsection{4-X-2: Skip List}\label{x-2-skip-list}

Implement a class template \texttt{cpppc::map} that satisfies the
\texttt{std::map} container interface based on a Skip List data
structure.

References:

\begin{itemize}
\tightlist
\item
  Specification of std::map:
  \url{http://en.cppreference.com/w/cpp/container/map}
\item
  \emph{Skip Lists: A Probabilistic Alternative to Balanced Trees}. W.
  Pugh, 1990. \url{http://cpppc.pub.lab.nm.ifi.lmu.de/docs/skiplist.pdf}
\end{itemize}

\end{document}
