\documentclass[]{article}
\usepackage{lmodern}
\usepackage{amssymb,amsmath}
\usepackage{ifxetex,ifluatex}
\usepackage{fixltx2e} % provides \textsubscript
\ifnum 0\ifxetex 1\fi\ifluatex 1\fi=0 % if pdftex
  \usepackage[T1]{fontenc}
  \usepackage[utf8]{inputenc}
\else % if luatex or xelatex
  \ifxetex
    \usepackage{mathspec}
  \else
    \usepackage{fontspec}
  \fi
  \defaultfontfeatures{Mapping=tex-text,Scale=MatchLowercase}
  \newcommand{\euro}{€}
\fi
% use upquote if available, for straight quotes in verbatim environments
\IfFileExists{upquote.sty}{\usepackage{upquote}}{}
% use microtype if available
\IfFileExists{microtype.sty}{%
\usepackage{microtype}
\UseMicrotypeSet[protrusion]{basicmath} % disable protrusion for tt fonts
}{}
\makeatletter
\@ifpackageloaded{hyperref}{}{%
\ifxetex
  \usepackage[setpagesize=false, % page size defined by xetex
              unicode=false, % unicode breaks when used with xetex
              xetex]{hyperref}
\else
  \usepackage[unicode=true]{hyperref}
\fi
}
\@ifpackageloaded{color}{
    \PassOptionsToPackage{usenames,dvipsnames}{color}
}{%
    \usepackage[usenames,dvipsnames]{color}
}
\makeatother
\hypersetup{breaklinks=true,
            bookmarks=true,
            pdfauthor={},
            pdftitle={Assignment 1: Object-Oriented C},
            colorlinks=true,
            citecolor=blue,
            urlcolor=blue,
            linkcolor=magenta,
            pdfborder={0 0 0}
            }
\urlstyle{same}  % don't use monospace font for urls
\usepackage{color}
\usepackage{fancyvrb}
\newcommand{\VerbBar}{|}
\newcommand{\VERB}{\Verb[commandchars=\\\{\}]}
\DefineVerbatimEnvironment{Highlighting}{Verbatim}{commandchars=\\\{\}}
% Add ',fontsize=\small' for more characters per line
\newenvironment{Shaded}{}{}
\newcommand{\KeywordTok}[1]{\textcolor[rgb]{0.00,0.44,0.13}{\textbf{{#1}}}}
\newcommand{\DataTypeTok}[1]{\textcolor[rgb]{0.56,0.13,0.00}{{#1}}}
\newcommand{\DecValTok}[1]{\textcolor[rgb]{0.25,0.63,0.44}{{#1}}}
\newcommand{\BaseNTok}[1]{\textcolor[rgb]{0.25,0.63,0.44}{{#1}}}
\newcommand{\FloatTok}[1]{\textcolor[rgb]{0.25,0.63,0.44}{{#1}}}
\newcommand{\ConstantTok}[1]{\textcolor[rgb]{0.53,0.00,0.00}{{#1}}}
\newcommand{\CharTok}[1]{\textcolor[rgb]{0.25,0.44,0.63}{{#1}}}
\newcommand{\SpecialCharTok}[1]{\textcolor[rgb]{0.25,0.44,0.63}{{#1}}}
\newcommand{\StringTok}[1]{\textcolor[rgb]{0.25,0.44,0.63}{{#1}}}
\newcommand{\VerbatimStringTok}[1]{\textcolor[rgb]{0.25,0.44,0.63}{{#1}}}
\newcommand{\SpecialStringTok}[1]{\textcolor[rgb]{0.73,0.40,0.53}{{#1}}}
\newcommand{\ImportTok}[1]{{#1}}
\newcommand{\CommentTok}[1]{\textcolor[rgb]{0.38,0.63,0.69}{\textit{{#1}}}}
\newcommand{\DocumentationTok}[1]{\textcolor[rgb]{0.73,0.13,0.13}{\textit{{#1}}}}
\newcommand{\AnnotationTok}[1]{\textcolor[rgb]{0.38,0.63,0.69}{\textbf{\textit{{#1}}}}}
\newcommand{\CommentVarTok}[1]{\textcolor[rgb]{0.38,0.63,0.69}{\textbf{\textit{{#1}}}}}
\newcommand{\OtherTok}[1]{\textcolor[rgb]{0.00,0.44,0.13}{{#1}}}
\newcommand{\FunctionTok}[1]{\textcolor[rgb]{0.02,0.16,0.49}{{#1}}}
\newcommand{\VariableTok}[1]{\textcolor[rgb]{0.10,0.09,0.49}{{#1}}}
\newcommand{\ControlFlowTok}[1]{\textcolor[rgb]{0.00,0.44,0.13}{\textbf{{#1}}}}
\newcommand{\OperatorTok}[1]{\textcolor[rgb]{0.40,0.40,0.40}{{#1}}}
\newcommand{\BuiltInTok}[1]{{#1}}
\newcommand{\ExtensionTok}[1]{{#1}}
\newcommand{\PreprocessorTok}[1]{\textcolor[rgb]{0.74,0.48,0.00}{{#1}}}
\newcommand{\AttributeTok}[1]{\textcolor[rgb]{0.49,0.56,0.16}{{#1}}}
\newcommand{\RegionMarkerTok}[1]{{#1}}
\newcommand{\InformationTok}[1]{\textcolor[rgb]{0.38,0.63,0.69}{\textbf{\textit{{#1}}}}}
\newcommand{\WarningTok}[1]{\textcolor[rgb]{0.38,0.63,0.69}{\textbf{\textit{{#1}}}}}
\newcommand{\AlertTok}[1]{\textcolor[rgb]{1.00,0.00,0.00}{\textbf{{#1}}}}
\newcommand{\ErrorTok}[1]{\textcolor[rgb]{1.00,0.00,0.00}{\textbf{{#1}}}}
\newcommand{\NormalTok}[1]{{#1}}
\usepackage{longtable,booktabs}
\setlength{\parindent}{0pt}
\setlength{\parskip}{6pt plus 2pt minus 1pt}
\setlength{\emergencystretch}{3em}  % prevent overfull lines
\providecommand{\tightlist}{%
  \setlength{\itemsep}{0pt}\setlength{\parskip}{0pt}}
\setcounter{secnumdepth}{0}

\title{Assignment 1: Object-Oriented C}
\date{}

% Redefines (sub)paragraphs to behave more like sections
\ifx\paragraph\undefined\else
\let\oldparagraph\paragraph
\renewcommand{\paragraph}[1]{\oldparagraph{#1}\mbox{}}
\fi
\ifx\subparagraph\undefined\else
\let\oldsubparagraph\subparagraph
\renewcommand{\subparagraph}[1]{\oldsubparagraph{#1}\mbox{}}
\fi

% HERGET
\usepackage{tcolorbox}
\tcbuselibrary{theorems, breakable, skins}
\newtcbtheorem{example}% environment name
              {Answer}% Title text
  {enhanced, % tcolorbox styles
  attach boxed title to top center={yshift=-2.5mm},
  colback=white, colframe=black, colbacktitle=white, coltitle=black,
  boxed title style={size=small,colframe=white},
  fonttitle=\bfseries,
  sharp corners=all,
  breakable
  }%
  {ex}% label prefix
 % END HERGET

\begin{document}
\maketitle

\textbf{C++ Programming Course, Winter Term 2016}

\section{0: Prerequisites}\label{prerequisites}

\subsection{0-1: GoogleTest}\label{googletest}

Install googletest in your home directory:

\begin{Shaded}
\begin{Highlighting}[]
\NormalTok{$ }\KeywordTok{mkdir} \NormalTok{~/build}
\NormalTok{$ }\KeywordTok{mkdir} \NormalTok{~/opt}
\NormalTok{$ }\KeywordTok{git} \NormalTok{clone https://github.com/google/googletest.git ~/build/googletest.git}
\NormalTok{$ }\KeywordTok{cd} \NormalTok{~/build/googletest.git}
\NormalTok{$ }\KeywordTok{cmake} \NormalTok{-DCMAKE_INSTALL_PREFIX=}\OtherTok{$HOME}\NormalTok{/opt/gtest}
\NormalTok{$ }\KeywordTok{make} \NormalTok{install}
\NormalTok{$ }\KeywordTok{echo} \StringTok{"export GTEST_BASE=}\OtherTok{$HOME}\StringTok{/opt/gtest"} \KeywordTok{>>} \NormalTok{~/.zshrc}
\end{Highlighting}
\end{Shaded}

\subsection{0-2: Tools}\label{tools}

On the test system (VM image or SSH account on server), \texttt{tmux}
and \texttt{vim} (actually \texttt{nvim}) are installed and configured
nicely. If you have no previous experience with these programs, make
yourself familiar with both to make your life easier.

\subsection{0-3: Notes on Running Tests}\label{notes-on-running-tests}

Source files for your implementation are located in
\texttt{assignment-01/solution}. To build and run your code and the test
suite, use:

\begin{Shaded}
\begin{Highlighting}[]
\NormalTok{$ }\KeywordTok{cd} \NormalTok{assignment-01/test}
\NormalTok{$ }\KeywordTok{make} \NormalTok{run}
\CommentTok{# same as}
\NormalTok{$ }\KeywordTok{make} \NormalTok{clean }\KeywordTok{;} \KeywordTok{make} \KeywordTok{&&} \KeywordTok{./testsuite.bin}
\end{Highlighting}
\end{Shaded}

To run individual test cases, specify the parameter
\texttt{-\/-gtest\_filter} when running the test suite:

\begin{Shaded}
\begin{Highlighting}[]
\NormalTok{$ }\KeywordTok{./testsuite.bin} \NormalTok{--gtest_filter=}\StringTok{"VectorTest.StandardConcept"}
\NormalTok{$ }\KeywordTok{./testsuite.bin} \NormalTok{--gtest_filter=}\StringTok{"StackTest.*"}
\end{Highlighting}
\end{Shaded}

Write basic implementations in the \texttt{solution/*.c} files first,
the build will fail in the linker stage otherwise because of undefined
references.

If you modify the test suites for debugging, remember to validate your
solution using the original test implementation before submitting.

\section{\texorpdfstring{1: Classes in C - ``Wax on, Wax
off''}{1: Classes in C - Wax on, Wax off}}\label{classes-in-c---wax-on-wax-off}

Remember the implementation of the \texttt{String} class presented in
the last course session. Let's implement some more.

We first define the \texttt{Standard} concept that specifies standard
operations which must be provided for any type:

\begin{longtable}[c]{@{}ll@{}}
\toprule
. & .\tabularnewline
\midrule
\endhead
\texttt{\textless{}T\textgreater{}\_\_new()\ \ \ \ \ \ \ -\textgreater{}\ o}
& creates a default instance of type \tabularnewline
\texttt{\textless{}T\textgreater{}\_\_copy(r)\ \ \ \ \ -\textgreater{}\ o}
& creates a new instance \texttt{o} as a copy of instance
\texttt{r}\tabularnewline
\texttt{\textless{}T\textgreater{}\_\_delete(o)} & destroys instance
\texttt{o} and releases its memory\tabularnewline
\texttt{\textless{}T\textgreater{}\_\_assign(o,r)\ -\textgreater{}\ o\textquotesingle{}}
& assigns instance \texttt{r} to instance \texttt{o}\tabularnewline
\texttt{\textless{}T\textgreater{}\_\_equals(o,r)\ -\textgreater{}\ b} &
\texttt{1} if \texttt{o} and \texttt{r} are equal, otherwise
\texttt{0}\tabularnewline
\bottomrule
\end{longtable}

Obviously, the functions implementing these operations for a specific
type are named according to the type, such as \texttt{wombat\_\_new} for
type \texttt{Wombat}.

\subsection{1-1: Vector Class in C}\label{vector-class-in-c}

Implement a class in C that satisfies the \texttt{Vector} concept with
the following semantics:

\begin{longtable}[c]{@{}ll@{}}
\toprule
. & Semantics \tabularnewline
\midrule
\endhead
\texttt{vector\_\_size(v)\ \ \ \ \ \ -\textgreater{}\ n} & number of
elements contained in vector \texttt{v}\tabularnewline
\texttt{vector\_\_empty(v)\ \ \ \ \ -\textgreater{}\ b} & \texttt{0} if
vector \texttt{v} has no elements, otherwise 1\tabularnewline
\texttt{vector\_\_begin(v)\ \ \ \ \ -\textgreater{}\ i} & iterator
\texttt{i} referencing the first element in \texttt{v}\tabularnewline
\texttt{vector\_\_end(v)\ \ \ \ \ \ \ -\textgreater{}\ i} & iterator
\texttt{i} referencing past the final element in
\texttt{v}\tabularnewline
\texttt{vector\_\_push\_back(v,e)} & append value \texttt{e} as last
element in vector \texttt{v}\tabularnewline
\texttt{vector\_\_pop\_back(v)\ \ -\textgreater{}\ e} & append value
\texttt{e} as last element in vector \texttt{v}\tabularnewline
\texttt{vector\_\_push\_front(v,e)} & append value \texttt{e} as first
element in vector \texttt{v}\tabularnewline
\texttt{vector\_\_pop\_front(v)\ -\textgreater{}\ e} & append value
\texttt{e} as first element in vector \texttt{v}\tabularnewline
\texttt{vector\_\_at(v,o)\ \ \ \ \ \ -\textgreater{}\ e} & returns value
\texttt{e} at offset \texttt{o} in vector \texttt{v}\tabularnewline
\bottomrule
\end{longtable}

Validate your implementation with the test specification in test suite
\texttt{VectorTest}.

Some thoughts on this:

\begin{itemize}
\tightlist
\item
  We use regular pointers as iterators. What if we implemented a list?
    \begin{example}{Iterators}{iter}
        A list which stores the adress of every element in the vector, would
        enable us to call every element very fast. Thus, it is quite complex to
        implement and update.
    \end{example}

\item
  Look up the interface of \texttt{std::vector} at
  http://cppreference.com. The vector concept in the STL does not
  specify some of the methods above. Which ones? Why? If the answer does
  not seem obvious to you, implement all methods first. You'll see.
    \begin{example}{Interface}{inter}
        The \texttt{std::vector} does not implement \texttt{push\_\_front} and
        \texttt{pop\_\_front}. It is very inefficient since it constantly
        shuffling the existing data up every time we do an insertion. A
        \texttt{deque} is designed to perform these operations.
    \end{example}
\end{itemize}

\subsection{1-2: Deque Class in C}\label{deque-class-in-c}

Implement a class in C that satisfies the \texttt{Deque} concept with
the following semantics:

\begin{longtable}[c]{@{}ll@{}}
\toprule
. & Semantics \tabularnewline
\midrule
\endhead
\texttt{deque\_\_size(d)\ \ \ \ \ \ -\textgreater{}\ n} & number of
elements contained in deque \texttt{d}\tabularnewline
\texttt{deque\_\_empty(d)\ \ \ \ \ -\textgreater{}\ b} & \texttt{0} if
deque \texttt{d} has no elements, otherwise 1\tabularnewline
\texttt{deque\_\_push\_back(d,e)} & append value \texttt{e} as last
element in deque \texttt{d}\tabularnewline
\texttt{deque\_\_push\_front(d,e)} & append value \texttt{e} as first
element in deque \texttt{d}\tabularnewline
\texttt{deque\_\_pop\_back(d)\ \ -\textgreater{}\ e} & append value
\texttt{e} as last element in deque \texttt{d}\tabularnewline
\texttt{deque\_\_pop\_front(d)\ -\textgreater{}\ e} & append value
\texttt{e} as first element in deque \texttt{d}\tabularnewline
\bottomrule
\end{longtable}

Validate your implementation with the test specification in test suite
\texttt{DequeTest}.

Keep in mind that you already have an implementation of the
\texttt{Vector} concept that might come handy. Note that it's not about
efficiency, yet. See bonus assignments below for more on this.

\subsection{1-3: Stack Class in C}\label{stack-class-in-c}

Implement a class in C that satisfies the \texttt{Stack} concept with
the following semantics:

\begin{longtable}[c]{@{}ll@{}}
\toprule
. & Semantics \tabularnewline
\midrule
\endhead
\texttt{stack\_\_size(s)\ \ \ -\textgreater{}\ n} & number of elements
contained in stack \texttt{s}\tabularnewline
\texttt{stack\_\_empty(s)\ \ -\textgreater{}\ b} & \texttt{0} if stack
\texttt{s} has no elements, otherwise 1\tabularnewline
\texttt{stack\_\_push(s,e)} & put value \texttt{e} on the
stack\tabularnewline
\texttt{stack\_\_pop(s)\ \ \ \ -\textgreater{}\ e} & remove value
\texttt{e} from the stack\tabularnewline
\bottomrule
\end{longtable}

Validate your implementation with the test specification in test suite
\texttt{StackTest}.

\subsection{1-4: Some Thoughts to Meditate
Over}\label{some-thoughts-to-meditate-over}

Is there a \texttt{stack} container in the STL? How about a
\texttt{deque} container? Go to \texttt{cppreference.com} and find out.
  \begin{example}{Stack \& Deque in CPP}{dest}
    Yes there is a \texttt{Stack} "is a container adapter that gives the
    programmer the functionality of a stack - specifically, a FILO (first-in,
    last-out) data structure."\\
    "std::deque (double-ended queue) is an indexed sequence container that
    allows fast insertion and deletion at both its beginning and its end. In
    addition, insertion and deletion at either end of a deque never invalidates
    pointers or references to the rest of the elements. "
  \end{example}

Now, see if you can find something special about their definitions. Can
you think of a reason for their differences?
  \begin{example}{Iterators}{iter}
        \texttt{Stack} is designed to perform FILO and \texttt{Deque} to insert
        in the front or back while keeping the memoryadress of the rest of the
        elements.
  \end{example}

\subsection{1-5: Reverse Algorithm}\label{reverse-algorithm}

This assignment is just about experimenting:

Implement the methods \texttt{vector\_\_reverse(v)},
\texttt{deque\_\_reverse(v)} and \texttt{stack\_\_reverse} that reverse
the element order of the respective container instance. For this:

\begin{itemize}
\tightlist
\item
  Try to formulate the semantics of the general reverse operation
  \texttt{\textless{}T\textgreater{}\_\_reverse} with respect to
  existing methods specified above.
\item
  Extend the test suites by tests of the reverse-methods.
\item
  Finally, implement the reverse-methods and run the test suite to check
  their correctness.
\end{itemize}

What do you notice about the similarities and differences in your
implementations of these functions?
  \begin{example}{Reverse}{rev}
    Since I use a quit easy but expensive way to implement the \texttt{reverse} functions those are completly similiar. I create a new object and insert the last element of the original object to the start of the new one.
  \end{example}

Now, have another look at the question in 1-4. If you did not find an
answer already, can you now think of a reason for the curious definition
of \texttt{std::stack} in the STL?

\subsection{1-6: Outlook to Upcoming Course
Session}\label{outlook-to-upcoming-course-session}

Assuming you had to implement the test cases yourself.

What feature are you missing in C that would drastically simplify the
implementation of the test cases for \texttt{Vector}, \texttt{Deque} and
\texttt{Stack}?
  \begin{example}{Outlook}{outlook}
    A templated function which uses a Type and/or a word which is inserted into a templated test-case.
    In C I implemented the concepts very similiar since the test-cases do not check whether the memory allocation (etc) is completly according to the concept.
  \end{example}

Hint: You might have noticed that the test suites are quite identical.
Why? Is there an advantage in this? Can you exploit this in C?

\section{1-X: Bonus Assignment: Efficient Memory
Management}\label{x-bonus-assignment-efficient-memory-management}

Your implementation of \texttt{Vector} and \texttt{Deque} from
assignments 1-1 and 1-2 should have correct semantics, but a basic
``just formally correct'' implementation is quite inefficient with
respect to memory management.

\begin{itemize}
\tightlist
\item
  Refactor \texttt{deque\_\_push\_front} and
  \texttt{deque\_\_pop\_front} such that worst-case complexity O(1) is
  maintained.
\item
  Refactor \texttt{vector\_\_push\_back} and
  \texttt{vector\_\_pop\_back} to improve their average case complexity.
\item
  Your modified implementation should still pass the test suites, of
  course.
\end{itemize}

As usual, you should have a close look at cppreference.com first.

\end{document}
